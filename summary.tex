\chapter*{Resumen}
\markboth{Resumen}{}
\addcontentsline{toc}{part}{Resumen}


\textit{This short overview of the thesis work is written in Spanish as required by the Spanish Government for thesis manuscripts in a foreign language.}

\selectlanguage{spanish}
La superfluidez, que es la habilidad de un fluido para fluir sin
viscosidad aparente, es una de las consecuencias m\'as impactantes de
la coherencia cu\'antica colectiva, con manifestaciones que van desde
la metaestabilidad de las supercorrientes en geometr\'ias
m\'ultiplemente conexas, la aparici\'on de v\'ortices cu\'anticos o la
existencia de una velocidad cr\'itica para flujos sin fricci\'on
cuando el fluido se dispersa alrededor de un
defecto. Tradicionalmente, la superfluidez se ha estudiado en sistemas
en equilibrio como $^4$He l\'iquido y gases at\'omicos
ultrafr\'ios. Los avances experimentales en \'optica no lineal, en
particular las microcavidades de polaritones han allanado el camino
para estudiar los fen\'omenos relacionados con la superfluidez en
condiciones de non equilibrio, en presencia de bombeo externo y
dissipaci\'on.

Igualmente interesante es la posibilidad de simular fases
topol\'ogicas de la materia, as\'i como los estados Hall cu\'anticos
enteros y fraccionarios, m\'as all\'a de los sistemas electr\'onicos
tradicionales. En particular, los experimentos de fot\'onica en
matrices de resonadores en presencia de bombeo y dissipaci\'on,
%\textit{driven-dissipative}, 
ofrecen un alto grado de control de los par\'ametros que describen el
sistema, as\'i como un acceso directo a los autoestados y el espectro
de energ\'ia.

En esta tesis estudiamos los efectos hidrodin\'amicos as\'i como las
propiedades topol\'ogicas de sistemas en presencia de bombeo y
dissipaci\'on,. En particular, analizamos el comportamiento de tipo
superfluido en microcavidades de polarit\'ones, as\'i como la
topolog\'ia del espacio de momentos de matrices de resonadores
acoplados.

Los polaritones excit\'onicos en microcavidades son cuasipart\'iculas
resultantes de la mezcla de excitones (pares electr\'on-hueco) y
fotones confinados dentro de microcavidades con pozos
cu\'anticos. Mientras que en los fluidos polarit\'onicos se han visto
comportamientos de coherencia colectiva, la conexi\'on entre las
distintas manifestaciones de comportamientos superfluidos es m\'as
compleja que en los sistemas en equilibrio. En esta tesis,
consideraremos tanto el caso de una configuraci\'on de un solo fluido
(estado de bombeo o de pump), como el caso de tres fluidos en el
r\'egimen de oscilaci\'on \'optica param\'etrica que emerge de la
dispersi\'on param\'etrica del estado pump a los estados de signal y
de idler. En ambos casos, miraremos la respuesta de la dispersi\'on de
los polaritones frente a un defecto est\'atico d\'ebil.

Para la configuraci\'on de un solo fluido, evaluaremos de forma
anal\'itica el arrastre ejercido por el fluido en el defecto. Para
velocidades bajas del fluido, la frequencia del pump clasifica el
espectro de excitaciones colectivas en tres categor\'ias: lineal,
difusivo y gapped. Vemos que tanto el r\'egimen lineal como el
difusivo comparten cualitativamente el cambio de comportamiento del
arrastre entre el r\'egimen subs\'onico y supers\'onico. La velocidad
cr\'itica donde se produce este cambio de comportamiento esta dada por
la velocidad del sonido en el r\'egimen lineal. En cambio, para los
espectros gapped encontramos que la velocidad cr\'itica sobrepasa la
velocidad del sonido. En todos los casos, vemos que el arrastre
residual en el r\'egimen subcr\'itico est\'a causado por la vida media
finida de los polaritones. Adem\'as, por debajo de la velocidad
cr\'itica, el arrastre var\'ia de forma lineal con la vida del
polarit\'on, de acuerdo con estudios num\'ericos previos.

El r\'egimen de oscilaci\'on \'optica param\'etrica presenta retos
adicionales relacionados con la presencia de tres fluidos
acoplados. Se ha demonstrado que la coherencia macrosc\'opica
espont\'anea que proviene del acoplamiento de fase entre el estado de
signal y el de idler es responsable de la metaestabilidad simult\'anea
del flujo cuantisado de ambos estados de signal y idler. Encontramos
que las modulaciones generadas por el defecto en cada fluido no vienen
solo determinadas por el anillo de dispersi\'on en el espacio de
momentos (Rayleigh ring) sino que cada componente tiene anillos
adicionales debidos a la interacci\'on con los otros componentes,
impuesta por procesos no lineales y param\'etricos. Se\~nalamos tres
factores que determinan cual de estos anillos tiene la mayor
influencia en la respuesta de cada fluido: la fuerza del acoplamiento
entre los tres fluidos, la resonancia del anillo con la dispersi\'on
energ\'etica de los polaritones y por \'ultimo la velocidad de grupo
del fluido junto con el tiempo de vida de los polaritones. Para las
condiciones t\'ipicas de dispersi\'on param\'etrica, el estado de pump
est\'a en el r\'egimen supercr\'itico, por lo tanto, tanto el estado
de signal como el estado de idler mostrar\'an la misma modulaci\'on
que se aprecia en el pump, con lo cual ninguno de los tres estados
manifiesta un comportamiento superfluido. Sin embargo, el signal
parece fluir sin fricci\'on en los experimentos, debido a que los tres
factores mencionados anteriormente se unen para reducir la amplitud de
las modulaciones por debajo de los niveles detectables.

Los sistemas en presencia de bombeo y dissipacion pueden mostrar
fen\'omenos interesantes aunque no haya interacciones entre las
part\'iculas constituyentes, debido a la topolog\'ia no trivial de las
bandas de energ\'ia. En la parte final de esta tesis, presentaremos
una propuesta realista para un experimento \'optico utilizando
matrices acopladas resonantes. Hemos estudiado de forma te\'orica el
modelo Harper-Hofstadter para sistemas con bombeo y decadimiento en
la presencia de una trampa arm\'onica d\'ebil. Si omitimos el bombeo
del l\'aser y las p\'erdidas, podremos interpretar los autovalores de
este sistema (bajo ciertas aproximaciones) como los niveles de Landau
en el espacio de momentos toroidal, donde la curvatura de Berry
act\'ua como un campo magn\'etico. Vemos que las principales
caracter\'isticas de estos autoestados pueden ser observadas en el
estado estacionario bajo un bombeo monocrom\'atico
coherente. Adem\'as, vemos que los niveles de Landau en el espacio de
momentos tienen caracter\'isticas claras en medidas espectrosc\'opicas
experimentales. Finalmente, discutiremos las propiedades geom\'etricas
de las bandas energ\'eticas y de las part\'iculas en campos
magn\'eticos.


\chapter*{Abstract}
\markboth{Abstract}{}
\addcontentsline{toc}{part}{Abstract}
\selectlanguage{english}

Superfluidity, the ability of a fluid to flow without apparent
viscosity, is one of the most striking consequences of collective
quantum coherence, with manifestations ranging from metastability of
supercurrents in multiply connected geometries to the appearance of
quantized vortices, or the existence of a critical velocity for
frictionless flow when scattering against a defect. While
traditionally investigated in equilibrium systems, like liquid
${}^4$He and ultracold atomic gases, experimental advances in
nonlinear optics, in particular regarding microcavity
exciton-polaritons, paved the way for studying superfluid-related
phenomena in a driven-dissipative framework.

Equally exciting is the possibility of realising topological phases of
matter, such as the integer or fractional quantum Hall states, outside
of traditional electronic systems. Photonics experiments in
driven-dissipative resonator arrays, in particular, offer a high
degree of controllability and tunability, as well as unprecedented
experimental access to the eigenstates and energy spectrum.

This thesis reports on hydrodynamic effects, as well as topological
properties, of driven-dissipative systems. In particular, we analyze
the superfluid-like behaviour of microcavity exciton-polaritons, as
well as the momentum-space topology of coupled resonator arrays.

Microcavity exciton-polaritons are quasiparticles resulting from the
mixing of excitons (bound electron-hole pairs) and photons confined
inside semiconductor microcavities. While polariton fluids have been
shown to display collective coherence, the connection between the
various manifestations of superfluid behaviour is more involved
compared to equilibrium systems. In this manuscipt, we consider both
the case of a single-fluid pump-only configuration, as well as the
three-fluid optical parametric oscillator regime that results from
parametric scattering of the pump to the signal and idler states. In
both cases, we look at the response of the moving polaritons
scattering against a weak static defect present in the microcavity.

For the single fluid, we evaluate analytically the drag exerted by the
fluid on the defect. For low fluid velocities, the pump frequency
classifies the collective excitation spectra in three different
categories: linear, diffusive-like and gapped. We show that both the
linear and diffusive-like cases share a qualitatively similar
crossover of the drag from the subsonic to the supersonic regime as a
function of the fluid velocity, with a critical velocity given by the
speed of sound found for the linear regime. In contrast, for gapped
spectra, we find that the critical velocity exceeds the speed of
sound. In all cases, we show that the residual drag in the subcritical
regime is caused by the nonequilibrium nature of the system. Also,
well below the critical velocity, the drag varies linearly with the
polariton lifetime, in agreement with previous numerical studies.

The optical parametric oscillator regime presents an additional
challenge, as one is dealing with three coupled fluids. The
spontaneous macroscopic coherence following the phase locking of the
signal and idler fluids has been already shown to be responsible for
their simultaneous quantized flow metastability. We find that the
modulations generated by the defect in each fluid are not only
determined by its associated scattering ring in momentum space, but
each component displays additional rings because of the cross-talk
with the other components imposed by nonlinear and parametric
processes. We single out three factors determining which one of these
rings has the biggest influence on each fluid response: the coupling
strength between the three fluids, the resonance of the ring with the
polariton dispersion, and the values of each fluid group velocity and
lifetime together establishing how far each modulation can propagate
from the defect.  For the typical conditions of parametric scattering,
the pump is in the supercritical regime, so the signal and idler will
show the modulations of the pump, meaning none of the three states
manifests superfluid behaviour. However, the signal appears to flow
without friction in the experimental study, because the three factors
mentioned above conspire to reduce the amplitude of its modulations
below currently detectable levels.

Driven-dissipative systems can show interesting phenomena also without
interactions, stemming from the nontrivial topology of their energy
bands. In the final part of this thesis, we present a realistic
proposal for an optical experiment using state-of-the-art coupled
resonator arrays. We study theoretically the driven-dissipative
Harper-Hofstadter model on a square lattice in the presence of a weak
harmonic trap. Without pumping and losses, the eigenstates of this
system can be understood, under certain approximations, as
momentum-space toroidal Landau levels, where the Berry curvature, a
geometrical property of an energy band, acts like a momentum-space
magnetic field. We show how key features of these eigenstates can be
observed in the steady-state of the driven-dissipative system under a
monochromatic coherent drive. We also show that momentum-space Landau
levels would have clear signatures in spectroscopic measurements in
such experiments, and we discuss the insights gained in this way into
geometrical energy bands and particles in magnetic fields.


%%% Local Variables:
%%% mode: latex
%%% TeX-master: "thesis_berceanu"
%%% End:
