\chapter*{Resumen}
\markboth{Resumen}{}
\addcontentsline{toc}{part}{Resumen}


\textit{This short overview of the thesis work is written in Spanish as required by the Spanish Government for thesis manuscripts in a foreign language.}

\selectlanguage{spanish}
Superfluidez, la capacidad de un fluido fluya sin aparente viscosidad,
es una de las consecuencias más llamativas del colectivo coherencia
cuántica, con manifestaciones que van desde la metaestabilidad de
supercorrientes en geometrías multiplican conectados a la aparición de
vórtices cuantificados, o la existencia de una velocidad crítica para
flujo sin fricción cuando la dispersión contra un defecto. Mientras
Tradicionalmente investigado en sistemas en equilibrio, como un
líquido ${}^4$He y gases atómicos ultrafríos, los avances
experimentales en óptica no lineal, en particular en relación con
microcavity excitón-polaritonas, allanó el camino para el estudio de
superfluido relacionada fenómenos en un marco impulsado por
disipativo.

Igualmente interesante es la posibilidad de realizar las fases
topológicas de la materia, tales como el número entero o estados Hall
cuántico fraccional, fuera de los sistemas electrónicos
tradicionales. experimentos en fotónica impulsado por el resonador
disipativo matrices, en particular, ofrecen un alto grado de
controlabilidad y capacidad de ajuste, así como sin precedentes el
acceso experimental a la energía y los estados propios del espectro.

Esta tesis informa sobre los efectos hidrodinámicos, así como
topológico propiedades de los sistemas, impulsado por disipativas. En
particular, se analiza el comportamiento superfluido-como de
microcavidad excitón-polaritonas, como así como la topología de
impulso en el espacio de las matrices de resonador acoplados.

Microcavity excitón-polaritonas son cuasi-partículas resultantes de la
(mezcla de excitones ligados pares electrón-hueco) y los fotones
confinado microcavidades semiconductoras interior. Mientras que los
fluidos de polariton han sido se muestra para mostrar la coherencia
colectiva, la conexión entre el diversas manifestaciones de la
conducta superfluido es más complicado en comparación con los sistemas
de equilibrio. En este manuscipt, consideramos tanto el caso de un
solo fluido de la bomba de sólo configuración, así como la tres fluido
régimen oscilador paramétrico óptico que resulta de dispersión
paramétrica de la bomba a los estados de señal y de inversión. En
ambos casos, nos fijamos en la respuesta de los polaritonas móviles
esparciendo contra un defecto estática débil presente en el
microcavity.

Para un único líquido, evaluamos analíticamente la resistencia
ejercida por el fluido sobre el defecto. Para bajas velocidades del
fluido, la frecuencia de bombeo clasifica los espectros de excitación
colectiva en tres diferentes Categorías: lineal, de difusión similar y
con huecos. Se demuestra que tanto el casos difusivo-como lineal y
comparten un cualitativamente similar cruce de la avenida de la
subsónica a supersónica el régimen como una en función de la velocidad
del fluido, con una velocidad crítica propuesta por el velocidad del
sonido se encontró para el régimen lineal. En contraste, para gapped
espectros, nos encontramos con que la velocidad crítica excede la
velocidad de sonar. En todos los casos, se muestra que la resistencia
residual en el subcrítico régimen es causada por la naturaleza de no
equilibrio del sistema. También, muy por debajo de la velocidad
crítica, la resistencia varía linealmente con la polariton curso de la
vida, de acuerdo con estudios anteriores numéricos.

El régimen de oscilador paramétrico óptico presenta un adicional de
reto, ya que uno está tratando con tres fluidos acoplados. los la
coherencia macroscópica espontánea tras el bloqueo de fase de la
fluidos de señal y de inversión ha sido ya demostrado ser responsable
de su metaestabilidad de flujo cuantificada simultánea. Nos
encontramos con que la modulaciones generados por el defecto en cada
fluido son no sólo determinada por su anillo de dispersión asociado en
el espacio de momentos, pero cada componente muestra anillos
adicionales debido a la diafonía con los otros componentes impuestas
por no lineal y paramétrico procesos. Nos destacamos tres factores que
determinan cuál de estos anillos tiene la mayor influencia en cada
respuesta de fluido: el acoplamiento fuerza entre los tres fluidos, la
resonancia del anillo con el dispersión polariton, y los valores de
cada velocidad de grupo de fluido y vida juntos establecer en qué
medida cada uno de modulación puede propagarse del defecto. Para las
condiciones típicas de dispersión paramétrico, La bomba está en el
régimen supercrítico, por lo que la señal y la rueda loca hará mostrar
las modulaciones de la bomba, es decir, ninguno de los tres estados se
manifiesta un comportamiento superfluido. Sin embargo, la señal parece
fluir sin fricción en el estudio experimental, debido a que los tres
factores mencionado anteriormente conspiran para reducir la amplitud
de sus modulaciones debajo de los niveles detectables actualmente.

sistemas-Driven disipativas pueden mostrar fenómenos interesantes
también sin interacciones, derivados de la topología no trivial de su
energía alzacuello. En la parte final de esta tesis, presentamos un
realista propuesta de un experimento óptico que utiliza el estado de
la técnica acoplada arrays resonador. Se estudia teóricamente el
disipador impulsado modelo de Harper-Hofstadter en una red cuadrada en
presencia de un débil trampa armónica. Sin bombeo y las pérdidas, los
estados propios de esta sistema puede entenderse, bajo ciertas
aproximaciones, como impulso-espacio toroidal niveles de Landau, donde
la curvatura Berry, una propiedad geométrica de una banda de energía,
actúa como un impulso en el espacio campo magnético. Se muestra cómo
las características clave de estos estados propios pueden ser
observado en el estado de equilibrio del sistema impulsado por
disipativo bajo una impulso coherente monocromática. También se
muestra que el impulso de Landau-espacio niveles tendrían firmas
claras en las mediciones espectroscópicas en tales experimentos, y se
discuten los conocimientos adquiridos de esta manera en bandas de
energía geométricos y partículas en campos magnéticos.


\chapter*{Abstract}
\markboth{Abstract}{}
\addcontentsline{toc}{part}{Abstract}
\selectlanguage{english}

Superfluidity, the ability of a fluid to flow without apparent
viscosity, is one of the most striking consequences of collective
quantum coherence, with manifestations ranging from metastability of
supercurrents in multiply connected geometries to the appearance of
quantized vortices, or the existence of a critical velocity for
frictionless flow when scattering against a defect. While
traditionally investigated in equilibrium systems, like liquid
${}^4$He and ultracold atomic gases, experimental advances in
nonlinear optics, in particular regarding microcavity
exciton-polaritons, paved the way for studying superfluid-related
phenomena in a driven-dissipative framework.

Equally exciting is the possibility of realising topological phases of
matter, such as the integer or fractional quantum Hall states, outside
of traditional electronic systems. Photonics experiments in
driven-dissipative resonator arrays, in particular, offer a high
degree of controllability and tunability, as well as unprecedented
experimental access to the eigenstates and energy spectrum.

This thesis reports on hydrodynamic effects, as well as topological
properties, of driven-dissipative systems. In particular, we analyze
the superfluid-like behaviour of microcavity exciton-polaritons, as
well as the momentum-space topology of coupled resonator arrays.

Microcavity exciton-polaritons are quasiparticles resulting from the
mixing of excitons (bound electron-hole pairs) and photons confined
inside semiconductor microcavities. While polariton fluids have been
shown to display collective coherence, the connection between the
various manifestations of superfluid behaviour is more involved
compared to equilibrium systems. In this manuscipt, we consider both
the case of a single-fluid pump-only configuration, as well as the
three-fluid optical parametric oscillator regime that results from
parametric scattering of the pump to the signal and idler states. In
both cases, we look at the response of the moving polaritons
scattering against a weak static defect present in the microcavity.

For the single fluid, we evaluate analytically the drag exerted by the
fluid on the defect. For low fluid velocities, the pump frequency
classifies the collective excitation spectra in three different
categories: linear, diffusive-like and gapped. We show that both the
linear and diffusive-like cases share a qualitatively similar
crossover of the drag from the subsonic to the supersonic regime as a
function of the fluid velocity, with a critical velocity given by the
speed of sound found for the linear regime. In contrast, for gapped
spectra, we find that the critical velocity exceeds the speed of
sound. In all cases, we show that the residual drag in the subcritical
regime is caused by the nonequilibrium nature of the system. Also,
well below the critical velocity, the drag varies linearly with the
polariton lifetime, in agreement with previous numerical studies.

The optical parametric oscillator regime presents an additional
challenge, as one is dealing with three coupled fluids. The
spontaneous macroscopic coherence following the phase locking of the
signal and idler fluids has been already shown to be responsible for
their simultaneous quantized flow metastability. We find that the
modulations generated by the defect in each fluid are not only
determined by its associated scattering ring in momentum space, but
each component displays additional rings because of the cross-talk
with the other components imposed by nonlinear and parametric
processes. We single out three factors determining which one of these
rings has the biggest influence on each fluid response: the coupling
strength between the three fluids, the resonance of the ring with the
polariton dispersion, and the values of each fluid group velocity and
lifetime together establishing how far each modulation can propagate
from the defect.  For the typical conditions of parametric scattering,
the pump is in the supercritical regime, so the signal and idler will
show the modulations of the pump, meaning none of the three states
manifests superfluid behaviour. However, the signal appears to flow
without friction in the experimental study, because the three factors
mentioned above conspire to reduce the amplitude of its modulations
below currently detectable levels.

Driven-dissipative systems can show interesting phenomena also without
interactions, stemming from the nontrivial topology of their energy
bands. In the final part of this thesis, we present a realistic
proposal for an optical experiment using state-of-the-art coupled
resonator arrays. We study theoretically the driven-dissipative
Harper-Hofstadter model on a square lattice in the presence of a weak
harmonic trap. Without pumping and losses, the eigenstates of this
system can be understood, under certain approximations, as
momentum-space toroidal Landau levels, where the Berry curvature, a
geometrical property of an energy band, acts like a momentum-space
magnetic field. We show how key features of these eigenstates can be
observed in the steady-state of the driven-dissipative system under a
monochromatic coherent drive. We also show that momentum-space Landau
levels would have clear signatures in spectroscopic measurements in
such experiments, and we discuss the insights gained in this way into
geometrical energy bands and particles in magnetic fields.


%%% Local Variables:
%%% mode: latex
%%% TeX-master: "thesis_berceanu"
%%% End:
