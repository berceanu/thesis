\chapter*{Resumen}
\markboth{Resumen}{}
\addcontentsline{toc}{part}{Resumen}
%\selectlanguage{spanish}

\textit{This short overview of the thesis work is written in Spanish as required by the Spanish Government for thesis manuscripts in a foreign language.}

Pellentesque dapibus suscipit ligula.  Donec posuere augue in quam.
Etiam vel tortor sodales tellus ultricies commodo.  Suspendisse
potenti.  Aenean in sem ac leo mollis blandit.  Donec neque quam,
dignissim in, mollis nec, sagittis eu, wisi.  Phasellus lacus.  Etiam
laoreet quam sed arcu.  Phasellus at dui in ligula mollis ultricies.
Integer placerat tristique nisl.  Praesent augue.  Fusce commodo.
Vestibulum convallis, lorem a tempus semper, dui dui euismod elit,
vitae placerat urna tortor vitae lacus.  Nullam libero mauris,
consequat quis, varius et, dictum id, arcu.  Mauris mollis tincidunt
felis.  Aliquam feugiat tellus ut neque.  Nulla facilisis, risus a
rhoncus fermentum, tellus tellus lacinia purus, et dictum nunc justo
sit amet elit.

Lorem ipsum dolor sit amet, consectetuer adipiscing elit.  Donec
hendrerit tempor tellus.  Donec pretium posuere tellus.  Proin quam
nisl, tincidunt et, mattis eget, convallis nec, purus.  Cum sociis
natoque penatibus et magnis dis parturient montes, nascetur ridiculus
mus.  Nulla posuere.  Donec vitae dolor.  Nullam tristique diam non
turpis.  Cras placerat accumsan nulla.  Nullam rutrum.  Nam vestibulum
accumsan nisl.

Aliquam erat volutpat.  Nunc eleifend leo vitae magna.  In id erat non
orci commodo lobortis.  Proin neque massa, cursus ut, gravida ut,
lobortis eget, lacus.  Sed diam.  Praesent fermentum tempor tellus.
Nullam tempus.  Mauris ac felis vel velit tristique imperdiet.  Donec
at pede.  Etiam vel neque nec dui dignissim bibendum.  Vivamus id
enim.  Phasellus neque orci, porta a, aliquet quis, semper a, massa.
Phasellus purus.  Pellentesque tristique imperdiet tortor.  Nam
euismod tellus id erat.


\chapter*{Summary}
\markboth{Summary}{}
\addcontentsline{toc}{part}{Summary}

We study the linear response of a coherently driven polariton fluid
in the pump-only configuration scattering against a point-like defect
and evaluate analytically the drag force exerted by the fluid on the
defect. When the system is excited near the bottom of the lower
polariton dispersion, the sign of the interaction-renormalised pump
detuning classifies the collective excitation spectra in three
different categories~[C. Ciuti and I. Carusotto, \emph{physica status
solidi (b)} \textbf{242}, 2224 (2005)]: linear for zero,
diffusive-like for positive, and gapped for negative detuning. We show
that both cases of zero and positive detuning share a qualitatively
similar crossover of the drag force from the subsonic to the
supersonic regime as a function of the fluid velocity, with a critical
velocity given by the speed of sound found for the linear regime. In
contrast, for gapped spectra, we find that the critical velocity
exceeds the speed of sound. In all cases, the residual drag force in
the subcritical regime depends on the polariton lifetime only. Also,
well below the critical velocity, the drag force varies linearly with
the polariton lifetime, in agreement with previous
work~[E. Cancellieri \emph{et al.}, \emph{Phys. Rev. B} \textbf{82},
224512 (2010)], where the drag was determined numerically for a
finite-size defect.
%

Superfluidity, the ability of a liquid or gas to flow with zero
viscosity, is one of the most remarkable implications of collective
quantum coherence. In equilibrium systems like liquid ${}^4$He and
ultracold atomic gases, superfluid behaviour conjugates diverse yet
related phenomena, such as persistency of metastable flow in multiply
connected geometries and the existence of a critical velocity for
frictionless flow when hitting a static defect.
%
The link between these different aspects of superfluid behaviour is
far less clear in driven-dissipative systems displaying collective
coherence, such as microcavity polaritons, which raises important
questions about their concurrency.
%
With a joint theoretical and experimental study, we show that the
scenario is particularly rich for polaritons driven in a three-fluid
collective coherent regime so-called optical parametric oscillator.
%
On the one hand, the spontaneous macroscopic coherence following the
phase locking of the signal and idler fluids has been shown to be
responsible for their simultaneous quantized flow metastability.
%
On the other hand, we show here that pump, signal and idler have
distinct responses when hitting a static defect; while the signal
displays hardly appreciable modulations, the ones appearing in pump
and idler are determined by their mutual coupling due to nonlinear and
parametric processes.
%%

We theoretically study the driven-dissipative Harper-Hofstadter model
on a 2D square lattice in the presence of a weak harmonic
trap. Without pumping and loss, the eigenstates of this system can be
understood, in certain limits, as momentum-space toroidal Landau
levels, where the Berry curvature, a geometrical property of an energy
band, acts like a momentum-space magnetic field. We show that key
features of these eigenstates can be observed in the steady-state of
the driven-dissipative system under a monochromatic coherent drive,
and present a realistic proposal for an optical experiment using
state-of-the-art coupled cavity arrays. We discuss how such
spectroscopic measurements may be used to probe effects associated
both with the off-diagonal elements of the matrix-valued Berry
connection and with the synthetic magnetic gauge.



%%% Local Variables:
%%% mode: latex
%%% TeX-master: "thesis_berceanu"
%%% End:
