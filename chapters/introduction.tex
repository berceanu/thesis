\chapter{Introduction}
% goal: 1500 words

\paragraph{original contribution}

\paragraph{Motivation for the layman}
\paragraph{Motivation for physicists}
\paragraph{Personal motivation}


\paragraph{Physicist view of the world}
% Planet with no ice paradigm

\paragraph{Connection to everyday world}
% Children on swing, swimming ducks, symmetry breaking

\paragraph{Applications}
% Quote about newborn child
% Faraday quote about electricity
% economic and social motivation of science
% polariton devices
% connection between superfluidity and superconductivity, high Tc
% fundamental understanding
%  - symmetry breaking: higgs boson, electroweak interaction (see layman book)


\paragraph{How birds fly together}
Consider a group of birds that show long-range ordered behaviour,
manifested by forming a flock under certain conditions. This behaviour
has been modeled in Ref.~\cite{Toner1995}, where a time step rule is
introduced, such that each individual bird in a group determines its
next direction on each time step by averaging the directions of its
neighbours and adding some random noise on top of that. It is shown
that, in the limit of the velocity magnitude going to zero, the model
reduces to the XY model in two dimensions, where the spin is
represented by the bird velocity. Since the 2D XY model does not
spontaneously break the symmetry at any finite temperature (as
justified by the Mermin-Wagner theorem), Ref.~\cite{Toner1995} goes on
to show that the appearance of the long-range ordered phase is a
direct consequence of nonequilibrium aspects of the model. In a
nutshell, the neighbours of one particular bird will be different at
different times, depending on the velocity field. This gives rise to a
time-dependent variable-ranged interaction, which can stabilize the
ordered phase.

\paragraph{Ducks on a lake emit Cherenkov radiation}
% Parallel between ducks on a lake and content of thesis

\paragraph{Thesis layout}
% This thesis consists of two parts, ..
% chapter descriptions


%%% Local Variables:
%%% mode: latex
%%% TeX-master: "../thesis_berceanu"
%%% End:
