\chapter{Introduction}

% TODO: insert chapter descriptions (see akhmerov)
% TODO: create abbreviations list
% GP: Gross-Pitaevskii
% TODO: mention \hbar = 1


\section{Preliminaries}
\label{sec:preliminary}

\subsection{Linear response theory}
\label{subsec:linear-response}

Let $\phi_0(\rv)$ be the steady state solution to the
GP equation in the time-independent trapping potential $U_0(\rv)$
%
\begin{equation}\label{eq:GP-atoms}
  H_{\text{GP}} \phi_0 = 0
\end{equation}
% 
with the GP Hamiltonian defined as\footnote{This section follows
  loosely the formalism presented in Ref.~\cite{9783540410478}.}
%
\begin{equation}\label{eq:GP-ham}
  H_{\text{GP}} \equiv -\frac{\nabla^2}{2m} + U_0 + gN_0\abs{\phi_0}^2 - \mu
\end{equation}
% 

This Hamiltonian describes a bosonic condensate of $N_0$ particles with
contact interactions quantified by $g$, and chemical potential $\mu$.

Now consider adding a small time-dependent perturbation on top of the
trap, giving $U(\rv,t)=U_0(\rv) + \delta U(\rv,t)$. We are interested
in the response of the condensate to this perturbation.  For weak
perturbations, we can perform a linearization of the GP equation
Eq.~\eqref{eq:GP-atoms} around the stationary solution $\phi_0$. This
approach is known in the literature as the ``linear response''
formalism.

The condensate wavefunction $\phi(\rv,t)$ evolves according to
%
\begin{equation}\label{eq:GP-atoms-evolution}
  i\partial_t \phi = \left[-\frac{\nabla^2}{2m} + U + gN_0 \abs{\phi}^2 - \mu\right] \phi
\end{equation}
% 
We assume a small deviation of the wavefunction from its initial
steady state
%
\begin{equation}\label{eq:ansatz-atoms}
  \phi(\rv,t) = \phi_0(\rv) + \delta\phi(\rv,t)
\end{equation}
% 
such that we can expand Eq.~\eqref{eq:GP-atoms-evolution} and keep only linear terms in $\delta\phi$ and $\delta U$. We get
%
\begin{equation}\label{eq:GP-atoms-lin}
  i\partial_t\delta\phi =  \left[-\frac{\nabla^2}{2m} + U_0-\mu\right]
  \delta\phi + 2 g N_0 \phi_0^{\star}\phi_0\delta\phi + gN_0\phi_0^2\delta\phi^{\star}
  +\delta U\phi_0
\end{equation}
% 
Note that Eq.~\eqref{eq:GP-atoms-lin} is not strictly linear due to
the coupling of $\delta\phi$ to $\delta\phi^{\star}$. To restore
linearity, we consider the functions $\delta\phi$ and
$\delta\phi^{\star}$ as being independent and write the linear system
%
\begin{equation}\label{eq:GP-atoms-system}
  i \partial_t \colvec{\delta\phi(\rv,t)}{\delta\phi^{\star}(\rv,t)}
  = \Lca_{GP} \colvec{\delta\phi(\rv,t)}{\delta\phi^{\star}(\rv,t)}
  + \colvec{S(\rv,t)}{-S^{\star}(\rv,t)}
\end{equation}
% 
where we have introduced the linear operator
%
\begin{equation}\label{eq:LGP}
  \Lca_{\text{GP}} = \mat{H_{\text{GP}}+gN_0\abs{\phi_0}^2}{g N_0 \phi_0^2}{-g N_0 \phi_0^{\star 2}}{-\left[H_{\text{GP}}+gN_0\abs{\phi_0}^2\right]^{\star}}
\end{equation}
% 
and the source term $S(\rv,t)=\delta U(\rv,t)\phi_0(\rv)$.  Note that
$\Lca_{\text{GP}}$ is not a Hermitian operator! The presence of the minus
sign in the second line is due to the fact that the particles obey
Bose statistics, as this is not present in the BCS theory.

The standard method of determining the time evolution of $\delta\phi$
is to expand it in the basis formed by the eigenvectors of
$\Lca_{\text{GP}}$.

There is a subtlety involved here, however, as in the general case
$\Lca_{\text{GP}}$ is not diagonalizable. It usually has too few
eigenvectors to span the whole functional space\footnote{We are
  referring to the $L^2 \times L^2$ space, with $L^2$ being the space
  of square-integrable complex functions.}.

The solution to this problem, detailed in Ref.~\cite{Castin_1998},
is to work in the subspace orthogonal to $\ket{\phi_0}$. It can be
shown that the component of the solution along $\ket{\phi_0}$ only
results in a change of phase of the total condensate wavefunction. For
the problems of interest in the rest of this manuscript, however, the
dimension of the space parallel to $\ket{\phi_0}$ that we must project
out is of order 1.

We now consider the eigenvalue equation for the operator $\Lca_{\text{GP}}$
%
\begin{equation}\label{eq:L-eigen}
  \Lca_{\text{GP}} \ket{\psi_k^R} = \epsilon_k \ket{\psi_k^R}
\end{equation}
% 
with $\ket{\psi_k^R}$ being the right eigenvector and $\epsilon_k$ its
corresponding eigenvalue
%
\begin{equation}\label{eq:psi-R}
  \ket{\psi_k^R} = \colvec{\ket{u_k}}{\ket{v_k}}
\end{equation}
% 
Similarly, we also introduce the left eigenvector, obeying
$\Lca_{\text{GP}}^{\dagger} \ket{\psi_k^L} = \epsilon_k^{\star} \ket{\psi_k^L}$,
and the orthonormality condition
$\braket{\psi_k^L}{\psi_q^R} = \delta_{k,q}$. 

Notice that $\Lca_{\text{GP}}$ and $\Lca_{\text{GP}}^{\dagger}$ are
connected by the unitary transformation\footnote{Note that this holds
  as long as the Hamiltonian $H_{\text{GP}}$ only contains real
  terms.}
%
\begin{equation}\label{eq:symmetry-1}
  \eta \Lca_{\text{GP}} \eta^{\dagger} = \Lca_{\text{GP}}^{\dagger}
\end{equation}
% 
where $\eta = \sigma_3 = \mat{1}{0}{0}{-1}$ is the third Pauli
matrix. We say that $\Lca_{\text{GP}}$ is $\eta$-Hermitian, meaning
that one can define a new scalar product
$\braket{\cdot}{\cdot}_{\eta} = \braket{\cdot}{\eta \cdot}$ with a
different signature, such that
$\braket{\cdot}{\Lca_{\text{GP}} \cdot}_{\eta} =
\braket{\Lca_{\text{GP}} \cdot}{\cdot}_{\eta}$. The operator $\eta$ is
usually called the metric operator, and, not suprisingly in our case,
it is the same as the one of the scalar Klein-Gordon equation. A
pseudo-Hermitian operator usually also posesses antilinear symmetries,
and as we will see below this is also the case for
$\Lca_{\text{GP}}$. Interestingly, for operators with a real spectrum,
it can be shown that one can define another metric $\eta_+$, which
guarantees a positive-definite inner product, or, in other words,
$\braket{\psi}{\psi}_{\eta_+} > 0$ (provided $\psi \neq 0$ of
course). This can be used to formulate a probabilistic quantum theory
for the new wave-functions $\psi^R$ and $\psi^L$. For the general
theory and properties of pseudo-Hermitian operators, we point the
interested reader to Ref.~\cite{MOSTAFAZADEH_2010}.


Using Eq.~\eqref{eq:symmetry-1}, we get the general form of the left
eigen-vectors as
%
\begin{equation}\label{eq:psi-L}
  \bra{\psi_k^L} = {\cal N}_k \left( \bra{u_k},\, -\bra{v_k} \right)
\end{equation}
% 
with ${\cal N}_k$ a normalization factor.  We can chose
${\cal N}_k = \pm 1$ and group the eigenvalues of $\Lca_{\text{GP}}$
into 3 families, according to the quantity
%
\begin{equation}\label{eq:norms}
  n_k = \braket{u_k}{u_k} - \braket{v_k}{v_k}
\end{equation}
% 
We therefore have: the ``$+$'' family, corresponding to $n_k=+1$, the
``$-$'' family, such that $n_k=-1$ and the ``$0$'' family, with
$n_k=0$.

% In the absence of the added weak perturbation, the time evolution of
% mode $k$ is given by $\exp(-i\epsilon_k t)$, from which we get the
% dynamical stability condition $\imag{(\epsilon_k)} \leq 0$ for all
% $k$. Dynamical stability is important because it insures that small
% perturbations will not induce the condensate wavefunction to evolve
% far from its steady state value.

We are now ready to write the completeness relation
%
\begin{equation}\label{eq:completeness}
  \sum_k \ket{\psi_k^R} \bra{\psi_k^L} = \mathbb{I}
\end{equation}
% 
Using Eq.~\eqref{eq:completeness}, we can decompose any column vector
as\footnote{The modes in the ``$0$'' family do not appear in this
  expansion as their components live in the space orthogonal to the one
  of our solution.}
%
\begin{multline}\label{eq:decomposition}
  \colvec{\ket{l_1}}{\ket{l_2}} = \sum_{k \in ``+" \mbox{\scriptsize family}} \left[\braket{u_k}{l_1} - \braket{v_k}{l_2}\right]\colvec{\ket{u_k}}{\ket{v_k}}\\
  + \sum_{k \in ``-" \mbox{\scriptsize family}} \left[\braket{v_k}{l_2} - \braket{u_k}{l_1}\right]\colvec{\ket{u_k}}{\ket{v_k}}
\end{multline}
% 
There is now a further symmetry of $\Lca_{\text{GP}}$ that we can
exploit in our problem, a sort of time-reversal ``spin''-flip (or
particle-hole) symmetry, namely
%
\begin{equation}\label{eq:symmetry-2}
   \Theta \Lca_{\text{GP}} \Theta^{\dagger} = -\Lca_{\text{GP}}
\end{equation}
%
where $\Theta = \sigma_1 \cal{K}$, with $\sigma_1 = \mat{0}{1}{1}{0}$
the first Pauli matrix and $\cal{K}$ the complex conjugation
antilinear operator.

This results in a duality between the ``$+$'' family with eigenvectors
$(u_k, v_k)$ and energy $\epsilon_k$ and the ``$-$'' family with
eigenvectors $(v_{-k}^{\star}, u_{-k}^{\star})$ and energy
$-\epsilon_{-k}^{\star}$.

We can now finally project Eq.~\eqref{eq:GP-atoms-system} onto the
eigenvectors of $\Lca_{\text{GP}}$. Using the above-mentioned duality and
Eq.~\eqref{eq:decomposition}, we get
%
\begin{equation}\label{eq:phi-column-expansion}
  \colvec{\delta\phi(\rv,t)}{\delta\phi^{\star}(\rv,t)} = \sum_{k \in ``+" \mbox{\scriptsize family}}
  b_k(t) \colvec{u_k(\rv)}{v_k(\rv)}
  + b_{-k}^{\star}(t) \colvec{v_{-k}^{\star}(\rv)}{u_{-k}^{\star}(\rv)}
\end{equation}
% 
with the complex amplitudes $b_k$ satisfying 
%
\begin{equation}\label{eq:amplitudes-bk}
  i \frac{d}{dt}b_k(t) = \epsilon_k b_k(t) + s_k(t)
\end{equation}
% 
where we introduced
%
\begin{equation}\label{eq:amplitudes-sk}
  s_k(t) = \left( \bra{u_k} ,\, -\bra{v_k} \right) \colvec{\ket{S(t)}}{-\ket{S^{\star}(t)}}
\end{equation}
% 

\subsection{Cherenkov emission}
\label{subsec:cherenkov}

We now turn to applying the formalism developed in
Sec.~\ref{subsec:linear-response} to a concrete physical example,
inspired by Ref.~\cite{Carusotto_2006}. Consider a 2-dimensional
Bose-Einstein atomic condensate in a state with well-defined momentum,
described by the plane wave
$\psi_0 \exp \left( i \bm{k_0} \rv - \omega_0 t \right)$. Using our
previous notation, this corresponds to
%
\begin{equation}\label{eq:atom-initial}
  \phi_0(\rv) = \psi_0 \exp \left( i \bm{k_0} \rv \right)
\end{equation}
% 
and a chemical potential $\mu = \omega_0$. Since we have no trap,
$U_0(\rv)=0$, and Eq.~\eqref{eq:GP-atoms} reduces to
%
\begin{equation}\label{eq:atom-MF}
  \omega_0 - \left( \frac{k_0^2}{2m} + g \rho_0 \right) = 0
\end{equation}
% 
where we have introduced the condensate density
$\rho_0 \equiv N_0 \abs{\psi_0}^2$.  

We now intoduce a weak perturbation in the form of a localized defect
potential $\delta U(\rv, t) = V_d(\rv)$, which can represent for
instance a laser spot depleting a small area of the condensate.

Using Eq.~\eqref{eq:atom-MF}, the GP Hamiltonian becomes
$H_{\text{GP}} = -\frac{\nabla^2}{2m} - \frac{k_0^2}{2m}$ and the
source term
$S(\rv) = \psi_0 V_d(\rv) \exp \left( i \bm{k_0} \rv \right)$. We now
get the linear operator for our problem in the form
%
\begin{equation}\label{eq:ourL}
  \Lca = \mat{-\frac{\nabla^2}{2m} - \frac{k_0^2}{2m} + g\rho_0}{g N_0 \psi_0^2 \exp \left( 2 i \bm{k_0} \rv \right)}{- g N_0 \psi_0^{\star 2} \exp \left( - 2 i \bm{k_0} \rv \right)}{-\left[ -\frac{\nabla^2}{2m} - \frac{k_0^2}{2m} + g\rho_0 \right]}
\end{equation}
% 
Notice that, due to the presence of the off-diagonal exponential
terms, $\Lca$ does not commute with the momentum operator, which is
the generator of the spatial translation group. Luckily, however, we
can restore translational invariance by a simple unitary
transformation, as shown below.

First, a brief reminder of basic quantum mechanics. Using the standard
commutation relations, one can show that, for a constant wavevector
$\bm{k_0}$, the unitary operator\footnote{The hat symbol denotes
  operators in the relevant Hilbert space.}
%
\begin{equation}\label{eq:trans-oper}
  \hat{T}(\bm{k_0}) = \exp \left( -i \bm{k_0} \hat{\rv} \right)
\end{equation}
% 
performs a translation in momentum space,
$\hat{T}(\bm{k_0}) \ket{\bm{k}} = \ket{\bm{k} - \bm{k_0}}$, with the
ket $\ket{\bm{k}}$ representing a single particle state with
wavevector $\bm{k}$ such that
$\hat{\bm{k}} \ket{\bm{k}} = \bm{k} \ket{\bm{k}}$. Using the
definitions above, one can easily obtain the commutator
%
\begin{equation}\label{eq:trans-commutator}
  \left[ \hat{\bm{k}},\, \hat{T}(\bm{k_0}) \right] = -\bm{k_0}\hat{T}(\bm{k_0}) 
\end{equation}
% 
This allows us to rewrite the following expressions
\begin{align}\label{eq:products}
  \begin{split}
    \hat{T}^{\dagger}(\bm{k_0})\hat{\bm{k}}\hat{T}(\bm{k_0})& = \hat{\bm{k}} - \bm{k_0}\hat{\mathbb{I}}\\
    \hat{T}(\bm{k_0})\hat{\bm{k}}\hat{T}^{\dagger}(\bm{k_0})& = \hat{\bm{k}} + \bm{k_0}\hat{\mathbb{I}}  
  \end{split}
\end{align}

We now recognize the two exponentials in Eq.~\eqref{eq:ourL} as being
the real-space representation of $\hat{T}^2(\bm{k_0})$ and its
hermitian conjugate. This motivates us to define the following unitary
operator
%
\begin{equation}\label{eq:ucal}
  \hat{{\cal T}}(\bm{k_0}) = \mat{\hat{T}(\bm{k_0})}{0}{0}{\hat{T}^{\dagger}(\bm{k_0})}
\end{equation}
% 
such that a unitary transformation of our operator $\Lca$ now restores translational
symmetry. Indeed, one can see that
%
\begin{equation}\label{eq:translated-L}
  \hat{{\cal T}} \hat{\Lca} \hat{{\cal T}}^{\dagger} = \mat{\frac{\left(\kop + \bm{k_0}\right)^2}{2m} - \frac{k_0^2}{2m} + g\rho_0}{g N_0 \psi_0^2}{- g N_0 \psi_0^{\star 2}}{-\left[\frac{\left(\kop - \bm{k_0}\right)^2}{2m} - \frac{k_0^2}{2m} + g\rho_0 \right]}
\end{equation}
% 
where we have make use of Eqs.~\eqref{eq:products} and we have written
$\hat{\Lca}$ in a base-independent representation.  In the subspace of
momentum eigenstates $\ket{\kv}$, we can write the (right-)eigenvalue
equation corresponding to Eq.~\eqref{eq:translated-L} as
%
\begin{equation}\label{eq:right-eigen}
  \Lca_{\text{GP}} [k] \colvec{U_\sigma(k)}{V_\sigma(k)} = \epsilon_\sigma(k) \colvec{U_\sigma(k)}{V_\sigma(k)}
\end{equation}
% 
where we have recovered the matrix representation of
Eq.~\eqref{eq:LGP}, and introduced the notation
$\omega_\sigma(\kv) = \bm{v}_0 \kv + \epsilon_\sigma(k)$. Here
$\sigma = \pm$ labels the 2 different eigenmodes and we defined
$\bm{v}_0 \equiv \frac{\kv_0}{m}$ as the speed of the condensate.

Notice that the $k=0$ mode has only one eigenvector. However, one can
safely exclude it as this mode does not imply energy or momentum
transport. Excluding the $k = 0$ point, one can then solve
Eq.~\eqref{eq:right-eigen} and obtain the celebrated Bogoliubov
excitation spectrum
%
\begin{equation}\label{eq:bogoliubov}
  \epsilon_\sigma(k) = \sigma \left[\frac{k^2}{2m}\left(\frac{k^2}{2m} + 2 g \rho_0 \right) \right]^{\frac{1}{2}}
\end{equation}
% 
with $\sigma = \pm$. Notice that the complex amplitudes $U_\sigma(k)$,
$V_\sigma(k)$ only depend on the absolute value of $\kv$, while the
(real) spectrum of Eq.~\eqref{eq:translated-L}, $\omega_\sigma(\kv)$
is the Bogoliubov spectrum with an additional Galilean boost
$\bm{v}_0 \kv$.

As can be seen from Eq.~\eqref{eq:symmetry-2}, the 2 eigen-families
$\sigma$ and $-\sigma$ are linked by a duality, stemming from the
$\cal{P} \cal{T}$ symmetry\footnote{This can be actually formally
  proven after defining the parity and time-reversal operators
  corresponding to our problem. For details, see
  Ref.~\cite{MOSTAFAZADEH_2010}.} of the Bogoliubov operator $\Lca$.
We therefore drop the subscript and make the convention that
$\left( U,\, V \right) \equiv \left( U_{+},\, V_{+}
\right)$. Furthermore, making use of Eq.~\eqref{eq:symmetry-1}, we can
act with $\sigma_3$ on the eigenstates of $\Lca_{\text{GP}}$ to obtain
the ones of $\Lca_{\text{GP}}^{\dagger}$. 

This finally leads us to a biorthonormal basis
$\left\{ \ket{\psi_\sigma^R(\kv)},\, \ket{\psi_\sigma^L(\kv)} \right\}$
with 4 basis vectors
%
\begin{equation}\label{eq:biorthobasis}
  \left\{ \colvec{U(k)}{V(k)}, \colvec{V^{\star}(k)}{U^{\star}(k)}, \colvec{U(k)}{-V(k)}, \colvec{-V^{\star}(k)}{U^{\star}(k)} \right\} \bigotimes \ket{\kv}
\end{equation}
% 
which fulfills the orthonormality condition
%
\begin{equation}\label{eq:binormality}
  \braket{\psi_{\sigma^{\prime}}^L(\kv^{\prime})}{\psi_\sigma^R(\kv)} = \delta_{\sigma,\sigma^{\prime}} \delta^2(\kv - \kv^{\prime})
\end{equation}
% 
provided of course that we normalize in such a way that
$\abs{U(k)}^2 - \abs{V(k)}^2 = 1$, and the completeness relation
%
\begin{equation}\label{eq:bicompleteness}
  \sum_{\sigma = \pm} \int d^2 \kv \ket{\psi_\sigma^R(\kv)}\bra{\psi_\sigma^L(\kv)} = 1
\end{equation}
% 
In this basis, the spectral decomposition of
Eq.~\eqref{eq:translated-L} is of the diagonal form
%
\begin{equation}\label{eq:bidecomposition}
  \hat{{\cal T}} \hat{\Lca} \hat{{\cal T}}^{\dagger} = \sum_{\sigma = \pm} \int d^2 \kv \omega_\sigma(\kv) \ket{\psi_\sigma^R(\kv)}\bra{\psi_\sigma^L(\kv)}
\end{equation}
% 
It is now a trivial matter to solve the linearized evolution equation
Eq.~\eqref{eq:GP-atoms-system} for our problem. In particular, for a
localized static defect potential $V_d(\rv) = g_V \delta^2(\rv)$, ...





%%% Local Variables:
%%% mode: latex
%%% TeX-master: "../thesis_berceanu"
%%% End:
