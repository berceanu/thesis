\chapter{Introduction}
% goal: 1500 words

%%
Microcavity polaritons, which are the quasiparticles resulting from
the coherent strong coupling between quantum well excitons and cavity
photons~\cite{9780199228942}, have unique mixed matter-light
properties that none of their constituents display on its own.
%%
Because of their energy dispersion and their strong nonlinearity
inherited from the excitonic components, polaritons continuously
injected by an external laser into a pump state with suitable
wavevector and energy can undergo coherent stimulated scattering into
two conjugate states~\cite{Ciuti_2000,Ciuti_2001,Ciuti_2003}, namely
the signal and the idler, a process known as optical parametric
oscillator (OPO) and described in detail in Sec.~\ref{sec:opo}.
%
Since their first realisation~\cite{Stevenson_2000, Savvidis_2000,
Savvidis_2000_b, Baumberg_2000, Saba_2001}, the interest in
microcavity optical parametric phenomena has involved several fields
of fundamental and applicative research~\cite{Edamatsu_2004,
Savasta_2005, Lanco_2006, Abbarchi_2011, Ardizzone_2012, Xie_2012,
Lecomte_2013}.
%
Recently, considerable resources have been invested in exploring the
fundamental properties of parametric processes, including the
possibility of macroscopic phase coherence and superfluid
behaviour~\cite{Carusotto_2013}.  In spite of the coherent nature of
the driving laser pump, the OPO process belongs to the class of
non-equilibrium phase transitions in which a $U(1)$ phase symmetry is
spontaneously broken~\cite{Wouters_2007}.  While the phase of the
pumped mode is locked to the incident laser, the phases of the signal
and idler are free to be simultaneously rotated in opposite
directions.  Because of this phase freedom, recent
experiments~\cite{Sanvitto_2010} have tested the OPO superfluid
properties by exploring the physics of the signal-idler order
parameter, demonstrating the existence and metastability of vortex
configurations. As the order parameter involves both signal and idler,
their phase windings have opposite
signs~\cite{Sanvitto_2010,Marchetti_2010,9783642241857}. Crucially,
this causes both OPO fluids to display quantized flow metastability
simultaneously.
%
While in equilibrium condensates different aspects of superfluidity
are typically closely related~\cite{Leggett_1999}, this is no longer
true in a non-equilibrium context such as for microcavity
polaritons~\cite{Carusotto_2013}.  In particular, those aspects of
superfluidity related to frictionless flow around defects are expected
to be much more involved in OPO condensates than for any other
investigated polariton condensates, such as for the case of incoherent
pumping~\cite{Kasprzak_2006,Wouters_2010}, and single-state resonantly
pumped microcavities~\cite{Amo_2009}.  Independent of the pumping
scheme, the driving and the polariton finite lifetime prompt questions
about the meaning of superfluid behaviour, when the spectrum of
collective excitations is complex rather than real, raising conceptual
interrogatives about the applicability of a Landau
criterion~\cite{Wouters_2010}.  However, an additional complexity
characterises the OPO regime, i.e., the simultaneous presence of three
oscillation frequencies and momenta for pump, signal and idler
correspondingly multiplies the number of collective excitation
branches~\cite{Wouters_2007}.  Note that from the experimental point
of view, pioneering experiments~\cite{Amo_2009_b} have observed a
ballistic nonspreading propagation of signal/idler polariton
wavepackets in a triggered-OPO configuration.  However, given the
complexity of the dynamics as well as the nonlinear interactions
involved in this time-dependent configuration~\cite{Szyma_ska_2010}, a
theoretical understanding of these observations is not yet complete.
%
Out of equilibrium quantum fluids such as polaritons in semiconductor
microcavities are the subject of an intensive study. Microcavity
polaritons, the quasiparticles presented in
Chapter~\ref{cha:polaritons}, result from the strong coupling of
cavity photons and quantum well excitons, and have the prerogative of
being easy to both manipulate, via an external laser, and detect, via
the light escaping from the cavity~\cite{9780199228942}. In
particular, resonant excitation allows the accurate tuning of the
fluid properties, such as its density and current. However, the
polariton lifetime being finite establishes the system as
intrinsically out of equilibrium: An external pump is needed to
continuously replenish the cavity of polaritons, that quickly, on a
scale of tens of picoseconds, escape.
%
Recently, the superfluid properties of a resonantly pumped polariton
quantum fluid in the pump-only configuration --- i.e., where no other
states aside the pump one are occupied by, for example, parametric
scattering (as in the OPO regime) --- have been actively investigated
both experimentally and theoretically~\cite{Carusotto_2004,
  Ciuti_2005, Amo_2009, Cancellieri_2010, Pigeon_2011, Amo_2011,
  Nardin_2011, Sanvitto_2011}. This pumping scheme, differently from
other cases, such as the resonant OPO regime (treated in
Chapter~\ref{cha:opo}) and the incoherent pumping scheme, creates a
polariton fluid that, inside the pump spot, is not characterised by a
free phase. In contrast, the phase of the pump state is locked to the
one of the external pumping laser. Nevertheless, it has been
predicted~\cite{Carusotto_2004,Ciuti_2005} and
observed~\cite{Amo_2009} that scattering can be suppressed below a
critical velocity, where the system displays superfluid behaviour,
similar to what has been predicted by the Landau criterion for
equilibrium superfluid condensates. Furthermore, a fixed phase clearly
prevents the formation of phase dislocations, such as vortices and
solitons. For this reason, it has been suggested~\cite{Pigeon_2011}
and experimentally realised~\cite{Amo_2011} that the defect can be
located just outside the pump spot, where the hydrodynamic nucleation
of vortices, vortex-antivortex pairs, arrays of vortices, and solitons
can be observed when the fluid collides with the extended
defect. Similarly, nucleation of vortices in the wake of the obstacle
has been observed in pulsed experiments
~\cite{Nardin_2011,Sanvitto_2011}.
%

In 1984, Michael Berry showed that the adiabatic evolution of energy
eigenfunctions with respect to a time-dependent Hamiltonian contains a
phase of geometrical origin, commonly known today by the name of
"Berry's phase" \cite{Berry1984}.  Although Berry's seminal paper gave
the example of a spinor's evolution under slowly changing magnetic
field, geometric phases \cite{shapere1989geometric} with similar
origin have been encountered in a plethora of physical contexts,
ranging from hydrodynamics to quantum field theory, the quantum Hall
effect and topological insulators.

In a condensed matter context, the eigenstates of electrons in a
periodic lattice can be labeled by a band index $n$ and the crystal
momentum $k$. The Berry-phase physics arising when adiabatically
transitioning between neighbouring wavevectors was appreciated only
recently. This phenomenon can explain properties such as electric
polarization, anomalous Hall conductivity or the quantization of
conductance in the integer quantum Hall effect.

Because the Berry phase in momentum space is not a gauge-invariant
quantity itself, one has to define the Berry connection and its curl,
the Berry curvature.  Drawing a parallel to classical electrodynamics,
the Berry phase plays the role of the Aharonov-Bohm phase, while the
Berry connection and curvature correspond to the vector potential and
magnetic field, respectively.  As shown by Thouless et
al. \cite{Thouless1982}, integrating the Berry curvature over the
Brillouin zone gives a topological invariant known as a Chern number,
which quantifies the Hall conductance.

Considering the Berry curvature to be an effective momentum space
magnetic field has been previously used in a semiclassical context,
for instance in order to explain the spin-Hall effect in p-type
semiconductors \cite{Murakami2003}.

Investigations along these lines are presently very active 
in both contexts of ultracold atoms and of polaritons. In particular,
Price and coworkers \cite{2014arXiv1403.6041P} have shown in an
ultracold atom context that the same interpretation of the Berry
curvature also works in a fully quantum picture. The model considered
in \cite{2014arXiv1403.6041P} is that of a particle described by the
Harper-Hofstadter(HH) Hamiltonian on a 2D lattice in the presence of
an external harmonic trapping potential.  The authors show that the
eigenspectrum of such a system can be understood as the Landau levels
of a particle moving in constant magnetic field on the surface of a
torus. This is a prime example of the physical consequences arising
from the topology of the Brillouin zone.

My work so far, part of which was done in an ongoing collaboration
between Universidad Autonoma de Madrid and the BEC Center in Trento,
focused on the superfluid properties of cavity exciton-polaritons
\cite{Berceanu2012,Tosi2011}. It would be therefore natural to use the
aquired expertise and search for a possible implementation of the
topological concepts introduced in \cite{2014arXiv1403.6041P} for
atomic systems to polariton systems.  In fact, energy bands with
nontrivial topology have been recently realised in such systems
\cite{Jacqmin2014}, using arrays of coupled micropillars. There are
several challenges to this approach, however, coming from the
driven-dissipative nature of such systems and the presence of optical
nonlinearities, not included in the HH model.


I focus on the implications of \cite{2014arXiv1403.6041P} for linear
optical systems. On one hand, I will consider systems of microcavity
polaritons on which I have acquired a good expertise in the last
years.  On the other hand, I will try to extend these ideas to novel
optical systems such as arrays of Si-based coupled ring resonators for
which there is an active experimental interest. With a proper design
of the resonator array \cite{Hafezi2013}, this latter system has been
demonstrated to be accurately described by the HH model with a strong
and uniform artificial gauge field piercing the lattice. As a further
addition, the harmonic trapping potential, which plays the role of
kinetic energy in momentum space, can easily be implemented as a
modulation of the resonator size.  I will start investigating how the
topological concepts are affected by the optical nonlinearities.


\paragraph{History of the subject}


\paragraph{original contribution}

\paragraph{Motivation for the layman}
\paragraph{Motivation for physicists}
\paragraph{Personal motivation}


\paragraph{Physicist view of the world}
% Planet with no ice paradigm

\paragraph{Connection to everyday world}
% Children on swing, swimming ducks, symmetry breaking

\paragraph{Applications}
% Quote about newborn child
% Faraday quote about electricity
% economic and social motivation of science
% polariton devices
% connection between superfluidity and superconductivity, high Tc
% fundamental understanding
%  - symmetry breaking: higgs boson, electroweak interaction (see layman book)

The source code is available online~\cite{SourceCode}.

\paragraph{How birds fly together}
Consider a group of birds displaying long-range ordered behaviour,
manifested by forming a flock under certain conditions. This behaviour
has been modeled in Ref.~\cite{Toner1995}, where a time step rule is
introduced, such that each individual bird in a group determines its
next direction on each time step by averaging the directions of its
neighbours and adding some random noise on top of that. It is shown
that, in the limit of the velocity going to zero, the model reduces to
the XY model in two dimensions, where the spin is represented by the
bird velocity. Since the 2D XY model does not spontaneously break the
symmetry at any finite temperature (as justified by the Mermin-Wagner
theorem), Ref.~\cite{Toner1995} goes on to show that the appearance of
the long-range ordered phase is a direct consequence of nonequilibrium
aspects of the model. In a nutshell, the neighbours of one particular
bird will be different at different times, depending on the velocity
field. This gives rise to a time-dependent variable-ranged
interaction, which can stabilize the ordered phase.
%
A planar system of electrons in a strong magnetic field is the
archetypal model for studying phenomena such as the integer and
fractional quantum Hall effects. With recent advances in creating
synthetic gauge fields, however, new horizons have opened up for
simulating such topological phases of matter also with neutral
particles, such as photons~\cite{hafezi2014synthetic} or ultracold
atoms~\cite{dalibardrmp2011, goldman_repprog_2014,
Goldman_arxiv_2015}.  Rather than simply replicating previous
measurements, experiments with synthetic gauge fields allow for
unprecedented access to properties such as the eigenstates or
eigenspectrum, while the tunability and controllability of these
experiments offer the prospect of simulating novel physics.


\paragraph{Ducks on a lake emit Cherenkov radiation}
% Parallel between ducks on a lake and content of thesis

\paragraph{Thesis layout}
% This thesis consists of two parts, ..
% chapter descriptions

A possible suggestion to link topology and polaritons is to go through
the works of Malpuech and Bardyn that have studied topological effects
in polariton lattices, where HH-like physics is obtained by breaking
time-reversal with a strong external magnetic field that lifts the
degeneracy between $\sigma_+$ and $\sigma_-$ polarized states of each
lattice site.




%%% Local Variables:
%%% mode: latex
%%% TeX-master: "../thesis_berceanu"
%%% End:
