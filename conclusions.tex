\chapter*{Conclusions}
\markboth{Conclusions}{}
\addcontentsline{toc}{part}{Conclusions}


\paragraph{Summary of the manuscript}
The main theme of this thesis was the study of well-established
phenomena, namely superfluidity and magnetism, in a novel context of
driven-dissipative photonic systems, which are subject to losses that
need to be compensated by continuous pumping. After introducing the
basic concepts in the setting of ultracold atomic gases in thermal
equilibrium, we looked at how the effects of driving and dissipation
changed this picture. In particular, we investigated the response of
microcavity exciton-polaritons scattering against a stationary defect,
with an eye on superfluid-like effects, and then turned to the
momentum-space magnetism of resonator arrays.

\paragraph{Polaritons}
In the case of polaritons, we looked at the single-fluid pump-only
regime, as well as the three coupled fluids of the optical parametric
oscillator regime created by the parametric scattering of the pump to
the signal and idler states. We found that none of the two regimes
displayed frictionless flow in the strict sense of the word. The
pump-only fluid showed a density modulation localized close to the
defect even in the ``superfluid regime'', resulting in a residual drag
force that is entirely due to the finite polariton lifetime. On the
other hand, we showed that the optical parametric oscillator regime,
in the optical limiter configuration, always displays propagating
density modulations in the pump, signal and idler, and therefore
violates the definition of frictionless flow in a stronger, conceptual
way. Futhermore, we determined two distinct types of threshold-like
behaviour of the drag force as a function of fluid velocity in the
pump-only case, one of which has no direct analog in equilibrium
weakly-interacting atomic gases. In the optical parametric oscillator
case, we singled out three factors which together determine the
amplitude of the density modulations, and explained why, for typical
experimental conditions, the signal state may appear superfluid.

\paragraph{Harper Hofstadter}
In the last part of the manuscript, we have discussed how
momentum-space Landau levels could be observed experimentally in
driven-dissipative photonic systems, by breaking time-reversal
invariance by means of a synthetic gauge field and creating
topologically nontrivial energy bands. We have presented a realistic
proposal for engineering these states by combining a harmonic trap
with an artificial magnetic field for photons in a two-dimensional
ring resonator array. An observation of momentum-space Landau levels
would be the first realisation of magnetism on a toroidal surface. We
have also demonstrated that the main properties of momentum-space
Landau levels may be observed spectroscopically for systems with
realistic experimental parameters. Furthermore, spectroscopic
measurements will be able to access the absolute energy of
eigenstates, measuring novel geometrical features of energy bands,
such as a contribution from the inter-band Berry connection. The
pumping and losses present in these experiments also open the way
towards studying analogue cyclotron orbits in momentum space.


\paragraph{Outlook}
In the future, more in-depth studies of polariton superfluidity,
especially regarding the optical parametric oscillator regime, are
needed. In experiments, one could look at multiply-connected
geometries, and on the theoretical side, use nonequilibrium Keldysh
field theory, which was already successfully employed in the case of
incoherently pumped systems, in order to rigorously compute the
superfluid and normal fractions. As far as the resonator arrays are
concerned, it would be interesting to include photon-photon
interaction in future studies, paving the way to fractional quantum
Hall physics in a driven-dissipative setting.


\chapter*{Conclusiones}
\markboth{Conclusiones}{}
\addcontentsline{toc}{part}{Conclusiones}


\textit{These closing conclusions are written in Spanish as required
  by the Spanish Government for thesis manuscripts in a foreign
  language.}

\selectlanguage{spanish}

El tema principal de esta tesis fue el estudio de la bien establecida
fenómenos, a saber, la superfluidez y el magnetismo, en un contexto de
novela sistemas fotónicos impulsadas por disipativa, que están sujetos
a las pérdidas que tienen que ser compensados ​​por bombeo
continuo. Después de la introducción de la conceptos básicos en la
configuración de los gases atómicos ultrafríos en térmica equilibrio,
nos fijamos en cómo los efectos de la conducción y disipación cambiado
esta imagen. En particular, se investigó la respuesta de microcavidad
excitón-polaritonas de dispersión contra un defecto estacionaria, con
la vista puesta en los efectos del superfluido por y, a continuación,
se volvió hacia el magnetismo impulso en el espacio de las matrices de
resonador.

En el caso de polaritonas, nos fijamos en el single-líquido de la
bomba de sólo régimen, así como los tres fluidos acoplados del régimen
OPO creados por la dispersión paramétrica de la bomba a la señal y la
rueda loca estados. Se encontró que ninguno de los dos regímenes de
muestra sin fricción fluir en el sentido estricto de la palabra. El de
sólo bomba de fluido mostró una modulación de la densidad localizada
cerca del defecto, incluso en el `` Régimen superfluido '', lo que
resulta en una fuerza de arrastre residual que es enteramente debido a
la vida útil polariton finito. Por otro lado, mostró que el régimen de
OPO, en la configuración limitador óptico, siempre muestra la
propagación de las modulaciones de densidad en la bomba, de la señal y
el piñón loco, y por lo tanto viola la definición de flujo sin
fricción de una forma más fuerte, conceptual. Futhermore, determinamos
dos distintos tipos de comportamiento umbral-como de la fuerza de
arrastre en función de la la velocidad del fluido en el caso de la
bomba de sólo, uno de los cuales no tiene directa analógico en
equilibrio débilmente interactuar gases atómicos. En el OPO caso,
hemos señalado tres factores que en conjunto determinan la amplitud de
las modulaciones de densidad, y explica por qué, por típica
condiciones experimentales, el estado de señal pueden aparecer
superfluido.

En la última parte del manuscrito, hemos hablado de cómo impulso en el
espacio niveles de Landau se pudieron observar experimentalmente en
sistemas fotónicos impulsadas por disipativa, mediante la ruptura de
la inversión del tiempo invariancia por medio de un campo de norma
sintético y la creación de topológicamente no triviales bandas de
energía. Hemos presentado un realista propuesta de ingeniería de estos
estados mediante la combinación de una trampa armónica con un campo
magnético artificial para fotones en una de dos dimensiones matriz de
anillo resonador. Una observación de cantidad de movimiento en el
espacio niveles de Landau sería la primera realización del magnetismo
sobre una superficie toroidal. Nosotros también han demostrado que las
propiedades principales de impulso-espacio niveles de Landau se pueden
observar espectroscópicamente para sistemas con parámetros
experimentales realistas. Además, espectroscópicas mediciones serán
capaces de acceder a la energía absoluta de estados propios, la
medición de las características geométricas de nuevas bandas de
energía, como una contribución de la conexión entre la baya de
banda. los bombeo y las pérdidas presentes en estos experimentos
también abren el camino al estudio de las órbitas ciclotrón analógicas
en el espacio de momentos.

En el futuro, los estudios más a fondo de la superfluidez polariton,
especialmente en relación con el régimen OPO, son necesarios. En los
experimentos, una podía mirar a geometrías multiconexos, y en el
teórico lado, utilizar la teoría de no equilibrio campo Keldysh, que
ya estaba empleado con éxito en el caso de los sistemas de bombeo
incoherentemente, en Para calcular rigurosamente las fracciones
superfluido y normales. Como En lo que se refiere a las matrices de
resonador, sería interesante incluir la interacción fotón-fotón en
estudios futuros, allanando el camino para fraccional física cuántica
Hall en un entorno impulsado por disipativo.

\selectlanguage{english}

%%% Local Variables:
%%% mode: latex
%%% TeX-master: "thesis_berceanu"
%%% End:
