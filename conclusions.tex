\chapter*{Conclusions}
\markboth{Conclusions}{}
\addcontentsline{toc}{part}{Conclusions}


To conclude, we have analysed the linear response to a weak defect of
resonantly pumped polaritons in the pump-only state, and we have been
able to determine two different kinds of threshold-like behaviours for
the drag force as a function of the fluid velocity. In the case of
either zero or positive pump detuning, one can continuously connect to
the case of equilibrium weakly interacting gases of
Chapter~\ref{cha:cold-gases}, where the drag displays a continuous
threshold with a critical velocity equal to the speed of
sound. However, for negative pump detuning, where the spectrum of
excitations is gapped, the drag shows a discontinuity with a critical
velocity larger than the speed of sound. In this sense, the case of
coherently driven microcavity polaritons in the pump-only
configuration displays a richer phenomenology than the case of
nonresonantly pumped polariton superfluids. We have also seen that the
absence of a long-range wake does not imply the absence of
dissipation, as a residual drag force due to the finite polariton
lifetime is always present in the system. In this sense, one can say
that we are not dealing with strict superfluidity.
%

To conclude, we have presented a joint theoretical and experimental
study of the superfluid properties of a nonequilibrium condensate of
polaritons in the so-called optical parametric oscillator
configuration by studying the scattering against a static defect.
%
We have found that, while the signal is basically free from
modulations, the pump and idler lock to the same response. We have
highlighted the role of the coupling between the OPO components due to
nonlinear and parametric processes. These are responsible for the
transfer of the spatial modulations from one component to the
other. This process is most visible in the clear spatial modulation
pattern that is induced by the nonsuperfluid pump onto the idler,
while the same modulations are only extremely weakly transferred into
the signal, because of its low characteristic wavevector, so much that
experimentally cannot be resolved.
%
The main features of the real- and momentum-space emission patterns
are understood in terms of Rayleigh scattering rings for each
component and a characteristic propagation length from the defect; the
rings are then transferred to the other components by nonlinear and
parametric processes.

Much interest has been recently devoted to aspects related to
algebraic order~\cite{Altman_2015,PhysRevX.5.041028} and superfluid
response~\cite{Keeling_2011_prl} in drive-dissipative polariton
condensates. Our theoretical and experimental results further stress
the complexities and richness involved when looking for superfluid
behavior in nonequilibrium multicomponent condensates such as the ones
obtained in the OPO regime.
%

In conclusion, we have shown that the observation of toroidal Landau
levels in momentum space is within experimental reach for
state-of-the-art driven-dissipative photonic systems. Our proposal
combines the recent realisation of the Harper-Hofstatder model in an
array of silicon-based coupled ring resonators in
Ref.~\cite{hafezi2013imaging}, with a harmonic potential, which could
be introduced through a spatial modulation of the resonator size. We
have presented numerical results to show that even for very small
lattices, in the presence of driving and strong dissipation, key
characteristics of the toroidal Landau levels can still be
extracted. This would be a first direct investigation of analogue
magnetic eigenstates in momentum space.

We have also emphasised that the proposed photonics experiment would
be able to highlight a momentum-space analog of the cyclotron motion
as well as to measure the energy shift due to the off-diagonal matrix
elements of the Berry connection, which, as these are inter-band
geometrical properties, are hard to access by other means. We have
also discussed how the spectroscopic measurements presented here are
sensitive to the specific synthetic magnetic gauge implemented in an
experiment.

Finally, an interesting outlook would be to include the effect of
photon-photon interactions in the model, as the degenerate ground
states predicted in~\cite{ozawa2014momhh} for a weakly-interacting
trapped Harper-Hofstadter model may lead to interesting nonlinear
dynamical features. In the longer run, when the synthetic gauge field
is combined with strong interactions, one can hope to observe the
hallmarks of fractional quantum Hall
physics~\cite{umucalilar2012fractional,hafezi2013non}.


\chapter*{Conclusiones}
\markboth{Conclusiones}{}
\addcontentsline{toc}{part}{Conclusiones}
%\selectlanguage{spanish}

\textit{These closing conclusions are written in Spanish as required
  by the Spanish Government for thesis manuscripts in a foreign
  language.}

Hendrerit tempor tellus.  Donec pretium posuere tellus.  Proin quam
nisl, tincidunt et, mattis eget, convallis nec, purus.  Cum sociis
natoque penatibus et magnis dis parturient montes, nascetur ridiculus
mus.  Nulla posuere.  Donec vitae dolor.  Nullam tristique diam non
turpis.  Cras placerat accumsan nulla.  Nullam rutrum.  Nam vestibulum
accumsan nisl.

Pellentesque dapibus suscipit ligula.  Donec posuere augue in quam.
Etiam vel tortor sodales tellus ultricies commodo.  Suspendisse
potenti.  Aenean in sem ac leo mollis blandit.  Donec neque quam,
dignissim in, mollis nec, sagittis eu, wisi.  Phasellus lacus.  Etiam
laoreet quam sed arcu.  Phasellus at dui in ligula mollis ultricies.
Integer placerat tristique nisl.  Praesent augue.  Fusce commodo.
Vestibulum convallis, lorem a tempus semper, dui dui euismod elit,
vitae placerat urna tortor vitae lacus.  Nullam libero mauris,
consequat quis, varius et, dictum id, arcu.  Mauris mollis tincidunt
felis.  Aliquam feugiat tellus ut neque.  Nulla facilisis, risus a
rhoncus fermentum, tellus tellus lacinia purus, et dictum nunc justo
sit amet elit.

Lorem ipsum dolor sit amet, consectetuer adipiscing elit.  Donec
hendrerit tempor tellus.  Donec pretium posuere tellus.  Proin quam
nisl, tincidunt et, mattis eget, convallis nec, purus.  Cum sociis
natoque penatibus et magnis dis parturient montes, nascetur ridiculus
mus.  Nulla posuere.  Donec vitae dolor.  Nullam tristique diam non
turpis.  Cras placerat accumsan nulla.  Nullam rutrum.  Nam vestibulum
accumsan nisl.


%%% Local Variables:
%%% mode: latex
%%% TeX-master: "thesis_berceanu"
%%% End:
