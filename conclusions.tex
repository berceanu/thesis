\chapter*{Conclusions}
\markboth{Conclusions}{}
\addcontentsline{toc}{part}{Conclusions}


\paragraph{Summary of the manuscript}
The main theme of this thesis was the study of well-established
phenomena, namely superfluidity and magnetism, in a novel context of
driven-dissipative photonic systems, which are subject to losses that
need to be compensated by continuous pumping. After introducing the
basic concepts in the setting of ultracold atomic gases in thermal
equilibrium, we looked at how the effects of driving and dissipation
changed this picture. In particular, we investigated the response of
microcavity exciton-polaritons scattering against a stationary defect,
with an eye on superfluid-like effects, and then turned to the
momentum-space magnetism of resonator arrays.

\paragraph{Polaritons}
In the case of polaritons, we looked at the single-fluid pump-only
regime, as well as the three coupled fluids of the optical parametric
oscillator regime created by the parametric scattering of the pump to
the signal and idler states. We found that none of the two regimes
displayed frictionless flow in the strict sense of the word. The
pump-only fluid showed a density modulation localized close to the
defect even in the ``superfluid regime'', resulting in a residual drag
force that is entirely due to the finite polariton lifetime. On the
other hand, we showed that the optical parametric oscillator regime,
in the optical limiter configuration, always displays propagating
density modulations in the pump, signal and idler, and therefore
violates the definition of frictionless flow in a stronger, conceptual
way. Futhermore, we determined two distinct types of threshold-like
behaviour of the drag force as a function of fluid velocity in the
pump-only case, one of which has no direct analog in equilibrium
weakly-interacting atomic gases. In the optical parametric oscillator
case, we singled out three factors which together determine the
amplitude of the density modulations, and explained why, for typical
experimental conditions, the signal state may appear superfluid.

\paragraph{Harper Hofstadter}
In the last part of the manuscript, we have discussed how
momentum-space Landau levels could be observed experimentally in
driven-dissipative photonic systems, by breaking time-reversal
invariance by means of a synthetic gauge field and creating
topologically nontrivial energy bands. We have presented a realistic
proposal for engineering these states by combining a harmonic trap
with an artificial magnetic field for photons in a two-dimensional
ring resonator array. An observation of momentum-space Landau levels
would be the first realisation of magnetism on a toroidal surface. We
have also demonstrated that the main properties of momentum-space
Landau levels may be observed spectroscopically for systems with
realistic experimental parameters. Furthermore, spectroscopic
measurements will be able to access the absolute energy of
eigenstates, measuring novel geometrical features of energy bands,
such as a contribution from the inter-band Berry connection. The
pumping and losses present in these experiments also open the way
towards studying analogue cyclotron orbits in momentum space.


\paragraph{Outlook}
In the future, more in-depth studies of polariton superfluidity,
especially regarding the optical parametric oscillator regime, are
needed. In experiments, one could look at multiply-connected
geometries, and on the theoretical side, use nonequilibrium Keldysh
field theory, which was already successfully employed in the case of
incoherently pumped systems, in order to rigorously compute the
superfluid and normal fractions. As far as the resonator arrays are
concerned, it would be interesting to include photon-photon
interaction in future studies, paving the way to fractional quantum
Hall physics in a driven-dissipative setting.


\chapter*{Conclusiones}
\markboth{Conclusiones}{}
\addcontentsline{toc}{part}{Conclusiones}


\textit{These closing conclusions are written in Spanish as required
  by the Spanish Government for thesis manuscripts in a foreign
  language.}

\selectlanguage{spanish}
El tema principal de esta tesis es el estudio de la superfluidez y
magnetismo en un nuevo contexto de los sitemas fot\'onicos en
presencia de bombeo y decadimiento. Estos sistemas est\'an sujetos a
p\'erdidas que necesitan ser compensadas con un bombeo
continuo. Despu\'es de introducir los conceptos b\'asicos para un
sistema de gases at\'omicos ultrafr\'ios en equilibrio t\'ermico vemos
c\'omo los efectos del bombeo y las p\'erdidas cambian el panorama. En
particular, investigamos la respuesta de la difusi\'on de los
polaritones excit\'onicos en microcavidades bajo la presencia de un
defecto estacionario, mirando los efectos de
superfluidez. Posteriormente miramos el magnetismo en el espacio de
momentos de matrices de resonadores.

En el caso de los polaritones consideramos el r\'egimen de un solo
fluido (estado de pump) y tambi\'en el caso de tres fluidos acoplados
en la configuraci\'on de oscilaci\'on \'optica param\'etrica creada
por la difusi\'on param\'etrica del pump en los estados signal y
idler. Encontramos que ninguno de los dos reg\'imenes manifiestan un
comportamiento de fluido sin fricci\'on en el sentido estricto de la
palabra. En el caso del fluido \'unico existe una modulaci\'on de la
densidad localizada cerca del defecto incluso en el r\'egimen
superfluido produciendo una fuerza de arrastre residual completamente
debida al tiempo de vida finito de los polaritones. Por otra parte,
vemos que el r\'egimen de oscilaci\'on \'optica param\'etrica en la
configuraci\'on del limitador \'optico siempre muestra modulaciones de
la densidad que se propaga en los estados de pump, signal y idler, y
por tanto viola la definici\'on de fluido sin fricci\'on de una forma
conceptual. Adem\'as determinamos dos tipos distintos de umbrales de
la fuerza de arrastre como funci\'on de la velocidad del fluido y del
pump, uno de ellos no tiene an\'alogo directo con gases at\'omicos en
equilibro d\'ebilmente interactuantes. En el caso del oscilador
\'optico param\'etrico se\~nalamos tres factores que juntos determinan
la amplitud de la modulaci\'on de la densidad, que explican la
aparente superfluidez del signal en condiciones experimentales
t\'ipicas.

En la \'ultima parte de esta tesis discutimos como los niveles de
Landau en el espacio de momentos pueden ser observados
experimentalmente en sistemas fot\'onicos en presencia de bombeo
externo y perditas, rompiendo la inversi\'on temporal por medio de un
campo magn\'etico artificial y creando bandas de energ\'ia
topol\'ogicamente no triviales. Presentamos una propuesta realista
para observar estos estados combinando una trampa arm\'onica con un
campo magn\'etico artificial para fotones en una matriz de anillos
resonantes bidimensional. Una observaci\'on de los niveles de Landau
en el espacio de momentos ser\'ia la primera observaci\'on del
magnetismo en una superficie toroidal. Adem\'as demostramos que las
propiedades principales de los niveles de Landau en el espacio de
momentos pueden ser observadas espectrosc\'opicamente en sistemas con
par\'ametros experimentales realistas. En el futuro, experimentos
espectrosc\'opicos podr\'an medir la energ\'ia absoluta de los
autoestados describiendo nuevas propiedades geom\'etricas de las
bandas de energ\'ia, como ocurre en el caso de la conexi\'on Berry
entre bandas. El bombeo y las p\'erdidas que hay en estos experimentos
abren una nueva v\'ia para estudiar \'orbitas an\'alogas a las del
ciclotr\'on en el espacio de momentos.

En el futuro se requerir\'an estudios m\'as a fondo de la superfluidez
de los polaritones, sobre todo en el r\'egimen de oscilaci\'on
\'optica param\'etrica. En los experimentos uno puede mirar
geometr\'ias m\'ultiplemente conexas. Desde el punto de vista de la
teor\'ia de Keldysh, basada en funciones de Green fuera del equilibro,
que ha sido utilizada de manera satisfactoria en el caso de sistemas
bombeados incoherentemente, para poder calcular de forma rigurosa las
fracciones tanto superfluida como normal. En el caso de las matrices
de resonadores podr\'ia ser interesante incluir la interacci\'on
fot\'on-fot\'on en futuros estudios, abriendo el camino en la f\'isica
del efecto Hall cu\'antico fraccionario en sistemas con bombeo y
dissipaci\'on.

\selectlanguage{english}

%%% Local Variables:
%%% mode: latex
%%% TeX-master: "thesis_berceanu"
%%% End:
